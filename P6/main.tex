\documentclass{article}
\usepackage[utf8]{inputenc} % Obsługa kodowania UTF-8
\usepackage[T1]{fontenc}    % Kodowanie czcionek T1
\usepackage{polski}         % Obsługa języka polskiego
\usepackage{babel}          % Wsparcie dla polskich reguł typograficznych
\usepackage{graphicx}       % Wymagane do wstawiania obrazów
\usepackage{amsmath}        % Obsługa zaawansowanych funkcji matematycznych
\usepackage[a4paper, margin=1in]{geometry}
\usepackage{listings}       % Obsługa wstawiania kodu źródłowego
\usepackage{xcolor}
\usepackage{listings}
\usepackage{verbatim}

%komenda do łatwiejszego wstawiania zdjęć
\newcommand*{\fg}[4][\textwidth]{
    \begin{figure}[!htb]
        \begin{center}
            \includegraphics[width=#1]{#2}
            \caption{#3}
            \label{rys:#4}
        \end{center}
    \end{figure}
}

\newcommand*{\Oznacz}[2]{
\ref{#1:#2} (s. \pageref{#1:#2})
}

\newcommand*{\OznaczZdjecie}[2][Rysunek]{
#1 \Oznacz{rys}{#2}
}
    
\newcommand*{\OznaczKod}[1]{
\Oznacz{lst}{#1}
}

\newcommand*{\ListingFile}[2]{
    \lstinputlisting[caption=#1, label={lst:#2}, language=CPP]{#2}
}

\definecolor{codegreen}{rgb}{0,0.6,0}
\definecolor{codegray}{rgb}{0.5,0.5,0.5}
\definecolor{codepurple}{rgb}{0.58,0,0.82}
\definecolor{backcolour}{rgb}{0.95,0.95,0.92}

% Definicja stylu "mystyle"
\lstdefinestyle{mystyle}{
	backgroundcolor=\color{backcolour},   
	commentstyle=\color{codegreen},
	keywordstyle=\color{blue},	%magenta
	numberstyle=\tiny\color{codegray},
	stringstyle=\color{codepurple},
	basicstyle=\ttfamily\footnotesize,
	breakatwhitespace=false,         
	breaklines=true,                 
	captionpos=b,                    
	keepspaces=true,                 
	numbers=left,                    
	numbersep=5pt,                  
	showspaces=false,                
	showstringspaces=false,
	showtabs=false,                  
	tabsize=2,
	literate=
  {á}{{\'a}}1 {é}{{\'e}}1 {í}{{\'i}}1 {ó}{{\'o}}1 {ú}{{\'u}}1
  {Á}{{\'A}}1 {É}{{\'E}}1 {Í}{{\'I}}1 {Ó}{{\'O}}1 {Ú}{{\'U}}1
  {à}{{\`a}}1 {è}{{\`e}}1 {ì}{{\`i}}1 {ò}{{\`o}}1 {ù}{{\`u}}1
  {À}{{\`A}}1 {È}{{\`E}}1 {Ì}{{\`I}}1 {Ò}{{\`O}}1 {Ù}{{\`U}}1
  {ä}{{\"a}}1 {ë}{{\"e}}1 {ï}{{\"i}}1 {ö}{{\"o}}1 {ü}{{\"u}}1
  {Ä}{{\"A}}1 {Ë}{{\"E}}1 {Ï}{{\"I}}1 {Ö}{{\"O}}1 {Ü}{{\"U}}1
  {â}{{\^a}}1 {ê}{{\^e}}1 {î}{{\^i}}1 {ô}{{\^o}}1 {û}{{\^u}}1
  {Â}{{\^A}}1 {Ê}{{\^E}}1 {Î}{{\^I}}1 {Ô}{{\^O}}1 {Û}{{\^U}}1
  {ã}{{\~a}}1 {ẽ}{{\~e}}1 {ĩ}{{\~i}}1 {õ}{{\~o}}1 {ũ}{{\~u}}1
  {Ã}{{\~A}}1 {Ẽ}{{\~E}}1 {Ĩ}{{\~I}}1 {Õ}{{\~O}}1 {Ũ}{{\~U}}1
  {œ}{{\oe}}1 {Œ}{{\OE}}1 {æ}{{\ae}}1 {Æ}{{\AE}}1 {ß}{{\ss}}1
  {ű}{{\H{u}}}1 {Ű}{{\H{U}}}1 {ő}{{\H{o}}}1 {Ő}{{\H{O}}}1
  {ç}{{\c c}}1 {Ç}{{\c C}}1 {ø}{{\o}}1 {Ø}{{\O}}1 {å}{{\r a}}1 {Å}{{\r A}}1
  {€}{{\euro}}1 {£}{{\pounds}}1 {«}{{\guillemotleft}}1
  {»}{{\guillemotright}}1 {ñ}{{\~n}}1 {Ñ}{{\~N}}1 {¿}{{?`}}1 {¡}{{!`}}1 
  {ą}{{\k{a}}}1 {ć}{{\'{c}}}1 {ę}{{\k{e}}}1 {ł}{{\l}}1 {ń}{{\'n}}1 
  {ó}{{\'o}}1 {ś}{{\'s}}1 {ź}{{\'z}}1 {ż}{{\.{z}}}1 
  {Ą}{{\k{A}}}1 {Ć}{{\'{C}}}1 {Ę}{{\k{E}}}1 {Ł}{{\L}}1 {Ń}{{\'N}}1
  {Ó}{{\'O}}1 {Ś}{{\'S}}1 {Ź}{{\'Z}}1 {Ż}{{\.{Z}}}1
}

\lstset{style=mystyle} % Deklaracja aktywnego stylu
%===========


\begin{document}
\sloppy
\begin{center}
    \small
        \renewcommand{\arraystretch}{1.3} % Zmniejszenie odstępów między wierszami
        \setlength{\tabcolsep}{20pt} % Zmniejszenie odstępów między kolumnami
        \begin{tabular}{|c|c|c|}
            \hline %chktex 44
            \multicolumn{3}{|c|}{\large Wydział Nauk Inżynieryjnych ANS w Nowym Sączu}   \\
            \multicolumn{3}{|c|}{\large }   \\
            \hline %chktex 44
            \multicolumn{3}{|c|}{\large Temat: P6} \\
            \hline %chktex 44
            \large Nazwisko i imię: & \large Ocena sprawozdania & \large Zaliczenie:\\
            \cline{2-3}
            \large Dominik Żuchowicz & & \\
            \hline
            \large Data wykonania ćwiczenia:&\multicolumn{2}{l|}{\large Grupa:}\\
            \large  17.03.2025 &\multicolumn{2}{l|}{\large L3}\\
            \hline
        \end{tabular}
    \end{center}

    \section*{Wprowadzenie}
    Współczesne systemy operacyjne i procesory oferują różnorodne mechanizmy synchronizacji, które pozwalają na bezpieczne współdzielenie zasobów między wieloma wątkami. Jednym z kluczowych elementów tych mechanizmów są operacje atomowe, które zapewniają integralność danych i eliminują problemy związane z wyścigami danych.

    \section*{Zadania}
        \begin{enumerate}
            
            \item Opisz czym są operacje atomowe.
            \item Dlaczego są tak przydatne w programowaniu wielowątkowym?
            \item Przetestuj KOD7/8 zastępując mutex operacjami atomowymi.
        \end{enumerate}
    \section*{Operacje atomowe}
    Operacje atomowe to takie operacje, które są wykonywane jako niepodzielne, co oznacza, że nie mogą zostać przerwane przez inne wątki w trakcie ich wykonywania. Są one realizowane sprzętowo przez procesor, co zapewnia ich integralność i spójność. Przykłady takich operacji to inkrementacja, dekrementacja, czy porównanie i zamiana wartości w pamięci.

    \section*{Przydatność w programowaniu wielowątkowym}
    Operacje atomowe są niezwykle przydatne w programowaniu wielowątkowym, ponieważ pozwalają uniknąć problemów związanych z wyścigami danych (ang. race conditions). Dzięki nim można bezpiecznie modyfikować współdzielone zasoby bez konieczności stosowania bardziej kosztownych mechanizmów synchronizacji, takich jak muteksy czy semafory. To z kolei prowadzi do poprawy wydajności i uproszczenia kodu w aplikacjach wielowątkowych.
    
    \section*{Testowanie KOD7/8}
    W ramach ćwiczenia przetestowano KOD7/8, zastępując mutex operacjami atomowymi. Poniżej przedstawiono fragmenty kodu, które ilustrują tę zmianę.

    \clearpage
    Przed zmianą:
    \fg{zdjecia/1.png}{Przed zmianą}{zdjecia-1}
     \\Po zmianie:
    \fg{zdjecia/2.png}{Po zmianie}{zdjecia-2}
    \clearpage
        

    \verbatiminput{kod/10_atomic.cpp}
    

    \section*{Wnioski}
    W wyniku przeprowadzonych testów stwierdzono, że zastosowanie operacji atomowych w miejsce mutexów znacząco poprawiło wydajność aplikacji. Operacje atomowe są szybsze i bardziej efektywne, co czyni je idealnym rozwiązaniem w przypadku prostych operacji na współdzielonych zasobach. Warto jednak pamiętać, że w bardziej złożonych scenariuszach, gdzie konieczne jest zarządzanie większą ilością danych lub bardziej skomplikowanymi strukturami, muteksy mogą być bardziej odpowiednie.

\end{document}
